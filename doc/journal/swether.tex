\subsection{Switched Ethernet Network}
\label{ssec:swether}

As a warmup to the more complicated case-studies presented in later sections,
we start with a high-level model of a redundant, switched ethernet network.
This network provides broadcast communication among a set of $n$ nodes and
provides redundancy through a set of $m$ independent switches. Figure
\ref{fig:swether-diagram} depicts the network for the choice of 3 nodes and 2
switches.

\begin{figure}
\begin{tikzpicture}[->,>=stealth',shorten >=1pt,auto,node distance=3cm,thick,
  main node/.style={circle,fill=blue!20,draw,font=\sffamily\Large\bfseries,
                   minimum size=15mm},
  switch node/.style={rectangle,draw,fill=black!20,minimum size=10mm},
  end node/.style={circle,draw,fill=blue!20,minimum size=8mm}]

  \node[main node] (A)               {$A$};
  \node[main node] (B) [right of=A]  {$B$};
  \node[main node] (C) [right of=B]  {$C$};

  \node[end node] (EA) [below of=A]   {$E_A$};
  \node[end node] (EB) [right of=EA]  {$E_B$};
  \node[end node] (EC) [right of=EB]  {$E_C$};

  \node[switch node] (S1) [below of=EA] {$S_1$};
  \node[switch node] (S2) [below of=EC] {$S_2$};

  % node to endpoints
  \draw[<->, thick] (A) to (EA);
  \draw[<->, thick] (B) to (EB);
  \draw[<->, thick] (C) to (EC);

  % endpoints to switches
  \draw[<->, thick] (EA) to (S1);
  \draw[<->, thick] (EA) to (S2);

  \draw[<->, thick] (EB) to (S1);
  \draw[<->, thick] (EB) to (S2);

  \draw[<->, thick] (EC) to (S1);
  \draw[<->, thick] (EC) to (S2);

\end{tikzpicture}
\caption{A switched ethernet network for 3 nodes and 2 switches}
\label{fig:swether-diagram}
\end{figure}

The network operates as follows. A node, say $A$, wants to broadcast a
message. It sends a message to an endpoint node $E_A$ that handles the
broadcast to each of the 2 switches $S_1$ and $S_2$. When a switch receives a
message it relays it to all the other endpoints on the network. The endpoints
are responsible for sorting out which message to eventually deliver to the
node. For simplicity we describe a network where endpoints deliver all
messages they receive from switches to their corresponding node.

To specify a network in LIMA we declare a function \y{mkSWEther} that takes as
parameters a number of nodes and a number of switches and returns a list of channel
input/output pairs. The channels are meant to be attached to user specified
nodes in a specific order. Internally, \y{mkSWEther} builds the endpoints and
switches as nodes and builds all the channels in between as well as those
pointing in and out of the network which will be returned.

Figure \ref{fig:swether-internal-chans} shows how the internal channels are
built. There is a unidirectional channel for each switch and each endpoint and
each other endpoint. They are stored in a particular order for ease of use in
attaching them to switches and endpoints.

\begin{figure}
\begin{lima}
-- generate the internal channels: [ [ (in_k_j, [out_1, ...]) ] ]
-- where in_k_j goes from endpoint j to switch k and out_1 .. out_{n-1} go
-- from switch k to the other endpoints (but not the j-th).
internalChans <-
  forM rm # \k ->       -- loop over switches
    forM rn # \j -> do  -- loop over endpoints
      in_k_j <- channel (printf "in_s%d_e%d" k j) typ
      let mkOChan i = do c <- channel (printf "out_s%v_e%v_e%v" k j i) typ
                         return (i,c)
      outs <- mapM mkOChan (bar j)
      return (in_k_j, outs)
\end{lima}
\caption{Declaration of internal channels}
\label{fig:swether-internal-chans}
\end{figure}

The switches are built from atoms having $n$ handler sub-atoms. Each handler
listens to a particular incoming channel (from one of the endpoints) and
whenever it sees a message there it broadcasts it out to all the other
endpoints. The declaration of switches is seen in Figure
\ref{fig:swether-switches}.

\begin{figure}
\begin{lima}
-- generate the switches:
-- each one listes on each incoming chan and broadcast to all outgoing chans
forM_ rm # \k ->
  atom (printf "sw%v" k) # do
    let myChans = internalChans !! k  -- :: [ (in_k_j, outs) ]_j
    forM_ rn # \j -> do
      let (myIn, myOuts) = myChans !! j
      atom (printf "handler_%v_%v" k j) # do
        cond # fullChannel (snd myIn)
        v <- readChannel (snd myIn)
        mapM_ ((`writeChannel` (v :: E Typ)) . fst . snd) myOuts
\end{lima}
\caption{Declaration of switches}
\label{fig:swether-switches}
\end{figure}

The endpoints are mostly similar to the switches: they listen to the channel
coming from the corresponding node and broadcast and received message to the
switches. However, in the opposite direction we have a problem. Each endpoint
must also listen to all the switches and decide what to do with the messages.
Listening is not a problem, we simply declare sub-atoms for each incoming
switch channel and setup them up to listen to the correct channel. But now we
need these sub-atoms to write each message they receive to the one outgoing
channel that points to the corresponding node. This is a problem because it
means we have multiple atoms writing to the same channel. This is not allowed
in LIMA by design. Instead we must buffer the sending of messages to the node.
In our case-study implementation we chose to buffer messages using a FIFO
queue of fixed length.
