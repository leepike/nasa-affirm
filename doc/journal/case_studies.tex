%% In the initial phase of our research, we have developed a prototype
%% Architectural Domain-Specific Language (ADSL) based workbench, translators of
%% ADSL specifications to formal and implementation models, and initial
%% proof-of-concept case studies. While the primary objective of ADSL workbench is
%% to synthesize a wide spectrum of both formal models and implementation models,
%% we only generate a subset of formal and implementation models in our initial work.

As mentioned in Section~\ref{sec:problem}, our ADSL is focused on the kinds of
systems found in avionics. Particularly, we limit ourselves to distributed
systems with the following characteristics: a finite, fixed, and usually small
number of nodes; a fixed number of communication channels between nodes;
possibly local and system-wide real-time constraints; and fault-tolerance
requirements and constraints.

We present here representative case-studies, each highlighting particular
aspects that the ADSL must handle. These case studies exercise and demonstrate
the breadth and expressiveness of our ADSL:

%% Thus far, we focus our efforts on capturing the following systems aspects:
%%  network architectural and distributed system attributes;
%%  Degree-of-Synchrony attributes (e.g. completely synchronous vs asynchronous vs period-synchronous) clock models;
%%  faults;
%%  functional properties and analysis.

\begin{itemize}
\item a high level model of a redundant, switched ethernet network
\item the Hybrid Oral Messages protocol~\cite{Lincoln-Rushby}, \OMH(1), which is a
    \emph{synchronous} Byzantine agreement protocol supporting a hybrid fault model;
\item and an \emph{asynchronous} ``push-button'' brake-by-wire~\cite{a320ibiza} case-study.
\end{itemize}

For each case-study, we first informally describe the protocol or system, then
we present its formalization in the ADSL. We focus on how the primitives and
library functions built on those primitives succinctly capture the model.
